\documentclass[a4paper,12pt]{article}

\begin{document}

\title{My First Document}
\author{Nikhil Ramesh}
\date{\today}
\maketitle


$1+2=3$

%  $$...$$ its own line

$$1+2=3$$

%  numbered displayed equation, use\begin{equation}...\end{equation}

\begin{equation}1+2=3\end{equation}

% Use\begin{eqnarray}...\end{eqnarray}to  write  equation  arrays  for  aseries of equations/inequalities.

\begin{eqnarray}
a & = & b + c \\
x & = & y - z
\end{eqnarray}

% For  unnumbered  equations  add  the  star  symbol*after  theequationoreqnarraycommand (i.e.  use{equation*}or{eqnarray*})

\begin{eqnarray*}
a & = & b + c \\
x & = & y - z
\end{eqnarray*}


% Powers are inserted using the hat^symbol.  For example,$n^2$

$n^2$


% If the power or index includes more than one character,  group them usingcurly brackets{...}, e.g.$b_{a-2}$producesba−2

$b_{a-2}$


% Fractions are inserted using\frac{numerator}{denominator}.

Fractions 
$$\frac{a}{3}$$

$$\frac{y}{\frac{3}{x}+b}$$\\


% square roots
square root $\sqrt{y^2}$\\

%sqaure roots 
More sqaure roots 
$\sqrt[x]{y^2}$ \\


%sum and integrals
Sum  $\sum_{x=1}^5 y^z$ \\

Integrals $\int_a^b f(x)$ \\

% GREEK LETTERS

  Greek letters
  
  $\alpha$
  $\beta$
  $\delta, \Delta$
  $\theta, \Theta$
  $\mu$
  $\pi, \Pi$
  $\sigma, \Sigma$
  $\phi, \Phi$
  $\psi, \Psi$
  $\omega, \Omega$




\end{document}